
\documentclass{beamer}

\AtBeginSection[]{
  \begin{frame}
  \vfill
  \centering
  \begin{beamercolorbox}[sep=8pt,center,shadow=true,rounded=true]{title}
    \usebeamerfont{title}\insertsectionhead\par%
  \end{beamercolorbox}
  \vfill
  \end{frame}
}



\begin{document}
\title{CFDS06 Project}   
\author{Felix Jobson} 
\date{\today} 

\frame{\titlepage} 

\frame{\frametitle{Table of contents}\tableofcontents} 




\section{Motivation} 
\frame{\frametitle{Motivation}
Evtl raus oder Proposal 
}



\section{Problem Description} 
\frame{\frametitle{Problem Description}
Siehe Proposal
}


\section{Data} 
\subsection{Sources and EDA}
\frame{\frametitle{Data sources}
\begin{itemize}
\item Quellen
\item Zeitraum
\item Welche Faktoren, Anzahl und Beispiele
\end{itemize}

}



\subsection{Missing Values}
\frame{\frametitle{Transforming to Growth Rates}
siehe Notebook
}




\frame{\frametitle{Preparation of the Data}
 \begin{itemize}
 \item Using the framework of supervised learning to work with time series. This approach is used in (Reference to Crystal-Ball)
 \item The orignal data is given in the form $(x_t, y_t)$, $t = 1 ... N$
 \item For every time step the outcome $y$ is mapped to predictor variables $x$ that are preceeding:  \begin{equation*}
 (x_{t-1}, y_t), ~ t = 2 ... N
 \end{equation*}
 \item Hence a model for supervised learning can be trained and used for prediction. 
 \end{itemize}
}


\frame{\frametitle{Impute missing values}
 
}


\section{Approach} 

\subsection{Models}


\frame{\frametitle{World Economic Outlook } 
\begin{itemize}
\item The International Monetary Fund publishes predictions of the GDP growth in its World Economic Outlook (WEO)
\item The IMF publishes the WEO twice a year in spring and in fall.
\item The prediction from the fall is used, as this  iscloser to the next year and therefore the prediction is more precise.
\end{itemize}
}



\frame{\frametitle{Classic Models} 
\begin{itemize}
\item Ordinary Least Squares
\begin{itemize}
\item The OLS regression is the most famous and basic model in econometrics. It has the following form: 
\begin{equation*}
y = x_1\beta_1 + x_2\beta_2 + ... x_N\beta_N + \beta_{N+1}
\end{equation*}
\end{itemize}



\item Autoregressive Integrated Moving Average
\begin{itemize}
\item The autoregressive integrated moving average ARIMA($p$, $d$, $q$) model is used in time series analysis. 
\item $X_t - \alpha_1X_{t-1} - ... - \alpha_pX_{t-p} =  \epsilon_t + \theta_1\epsilon_{t-1} + ... + \theta_q\epsilon_{t-q}$ 
\item Here $\alpha _{i}$ are the parameters of the autoregressive part of the model, $\theta _{i}$ are the parameters of the moving average part, $d$ is the degree of differencing and $\epsilon _{t}$ are error terms.  
\end{itemize}

\end{itemize}
}



\frame{\frametitle{Machine Learning Models I} 
\begin{itemize}
\item Least Absolute Shrinkage and Selection Operator
\begin{itemize}
\item The LASSO is a penalized version of the OLS: 
\begin{equation*}
\min_{\beta} { ||X \beta - y||_2 ^ 2 + \alpha ||\beta||_1 }
\end{equation*}
\end{itemize}



\item Support Vector Regression
\begin{itemize}
\item The SVR is an adapted version of a SVM for regression problems and tries to solve the optimization problem: 
  \begin{equation*}
\min_{\beta} { \frac{1}{2} || \beta ||_2 ^ 2 } \text{, subject to  } || X \beta - y || < \varepsilon
\end{equation*}
\end{itemize}

\end{itemize}
}



\frame{\frametitle{Machine Learning Models II} 
\begin{itemize}
\item Regression Tree
\begin{itemize}

\item Binary tree that groups data with similar vaules into the same leaf. 
\item This is done to minimize the overall loss: 
\begin{equation*}
L(...
\end{equation*}


%\item A regression tree is a rule based model.
%\item Based on a binary tree. 



\end{itemize}



\item Gradient Booster
\begin{itemize}
\item Ensemble of the from \begin{equation*}
f(x) = \sum_{i = 1}^N f_i(x)
\end{equation*} where $f_i$ are weak learners, most of the time tree based models. 
\item Are called gradient booster because of the way the model is trained.
\end{itemize}


\end{itemize}
}







\frame{\frametitle{Deep Learning} 


\begin{itemize}
\item  Reccurent Neural Network


\begin{itemize}
\item A RNN is a deep neural network that is designed to handle sequential data. 
\item A RNN cell is defined as:
\begin{equation*}
h_t = \sigma(W_{ih} x_t + b_{ih} + W_{hh} h_{(t-1)} + b_{hh})
\end{equation*}
\item There are also more sophisticated approched like the LSTM (Long short-term memory). 
\end{itemize}
\end{itemize}


}





\subsection{Training}
\frame{\frametitle{Training} 
Notebook. 
}


\subsection{Evaluation}
\frame{\frametitle{Evaluation} 
Noch Schreiben. Erkärung des signifikanz tests
}



\section{Results} 
\subsection{Training on single countries}
\frame{\frametitle{Results - Training on single countries}
Tabelle hier
}

\frame{\frametitle{Results - Training on single countries} 
\begin{figure}[htbp] 
\centering
\includegraphics[scale=0.65]{Germany.png}   
      % \caption{Simulationsumgebung RoboDK.}
\end{figure}
}


\subsection{Training on whole dataset countries}
\frame{\frametitle{Results - Training on single countries}
Tabelle hier
}

\frame{\frametitle{Results - Training on single countries} 
\begin{figure}[htbp] 
\centering
\includegraphics[scale=0.65]{Germany.png}   
      % \caption{Simulationsumgebung RoboDK.}
\end{figure}
}


\subsection{Test for statistical significance}
\frame{\frametitle{Test for statistical significance}
\begin{figure}[htbp] 
\centering
\includegraphics[scale=0.65]{result.png}   
      % \caption{Simulationsumgebung RoboDK.}
\end{figure}
}



\section{Conclusion} 
\frame{\frametitle{Conclusion}

}



\end{document}
