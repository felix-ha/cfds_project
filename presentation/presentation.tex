
\documentclass{beamer}

\AtBeginSection[]{
  \begin{frame}
  \vfill
  \centering
  \begin{beamercolorbox}[sep=8pt,center,shadow=true,rounded=true]{title}
    \usebeamerfont{title}\insertsectionhead\par%
  \end{beamercolorbox}
  \vfill
  \end{frame}
}



\begin{document}
\title{CFDS06 Project 

\ \\

 Macroeconomic forecasting: Can machine learning methods outperform traditional approaches?}   
\author{Felix Jobson} 
\date{\today} 

\frame{\titlepage} 

\frame{\frametitle{Table of contents}\tableofcontents} 




\section{Motivation} 
\frame{\frametitle{Motivation}
Evtl raus oder Proposal 
}



\section{Problem Description} 
\frame{\frametitle{Problem Description}
\begin{itemize}
\item The research question of the project is the capability of machine learning models to predict the growth of an economy and compare the result with traditional methods of forecasting.

\ \\

\item The dependent variable is the growth rate of the gross domestic product (GDP). This is the
objective of the learning and prediction task. The independent variables are several
macroeconomic factors.

\ \\

\item The baseline models are classical econometric methods and the World Economic Outlook of the International Monetray Fund.



\end{itemize}
}

\section{Data} 
\subsection{Sources and Overview}
\frame{\frametitle{Data sources}
\begin{itemize}

\item Sources
\begin{itemize}
\item International Monetary Fund (IMF)
\item Organisation for Economic Co-operation and Development (OECD)
\end{itemize}

\ \\

\item Time Period: 1980 - 2017
\begin{itemize}
\item Training Set: 1980 - 2004
\item Validation Set: 2005 - 2010
\item Test Set: 2011 - 2017 
\end{itemize}

\ \\

\item Countries:
\begin{itemize}
\item Initially 189
\item After clearning 46
\end{itemize}

 


\end{itemize}
}


\subsection{Variables}
\frame{\frametitle{Variables}

\begin{itemize}

\item Number of macroeconomic factors used: 
\begin{itemize}
\item Initial: 41
\item After cleaning 15
\end{itemize}


\ \\

\item Examples of used varibles 
\begin{itemize}
\item Inflation
\item Unemployment rate
\item Material consumption
\item Working age population
\item Fertility rates

\end{itemize}


\end{itemize}


}




\subsection{Missing Values}
\frame{\frametitle{Transforming to Growth Rates}
\begin{itemize}
\item Because the variable have different absolute values, growth rates are used. 

\ \\

\item In order to receive the same magnitue for an increase as well as a decrease a logarithmic tranformation is used: 

\ \\

\item 
\begin{equation*}
\hat{x}_{i} = \ln(\frac{x_{i}}{x_{i-1}} + |\min_i(x_{i})| + 0.001)
\end{equation*}
\end{itemize}
}




\frame{\frametitle{Preparation of the Data}
 \begin{itemize}
\item Using the framework of supervised learning to work with time series.

\ \\

\item The orignal data is given in the form $(x_t, y_t)$, $t = 1 ... N$

\ \\

\item For every time step the outcome $y$ is mapped to predictor variables $x$ that are preceeding:  \begin{equation*}
(x_{t-1}, y_t), ~ t = 2 ... N
 \end{equation*}
 \item Hence a model for supervised learning can be trained and used for prediction. 
 \end{itemize}
}


\frame{\frametitle{Impute missing values}
 \begin{itemize}
 \item Only countries with less than 50 \% missing values an are used. Then the top 15 filled variables are selected.  
 
 \ \\

 \item To use time series with missing data at all, an imputing strategy is used: \textit{k-nearest neighbors} 
 
\ \\

 
 \item Each samples missing values are imputed using the mean value from n nearest neighbors found in the training set.
 
 \ \\
 
 \item Important: Fit on the training set and then appy imputation on the validation and test set. 
 


 \end{itemize}
}


\section{Approach} 

\subsection{Models}


\frame{\frametitle{World Economic Outlook } 
\begin{itemize}
\item The International Monetary Fund publishes predictions of the GDP growth in its World Economic Outlook (WEO)

\ \\

\item The IMF publishes the WEO twice a year in spring and in fall.

\ \\

\item The prediction from the fall is used, as this  iscloser to the next year and therefore the prediction is more precise.
\end{itemize}
}



\frame{\frametitle{Classic Models} 
\begin{itemize}
\item Ordinary Least Squares
\begin{itemize}
\item The OLS regression is the most famous and basic model in econometrics. It has the following form: 
\begin{equation*}
y = x_1\beta_1 + x_2\beta_2 + ... x_N\beta_N + \beta_{N+1}
\end{equation*}
\end{itemize}

\ \\

\item Autoregressive Integrated Moving Average
\begin{itemize}
\item The autoregressive integrated moving average ARIMA($p$, $d$, $q$) model is used in time series analysis. 
\item $X_t - \alpha_1X_{t-1} - ... - \alpha_pX_{t-p} =  \epsilon_t + \theta_1\epsilon_{t-1} + ... + \theta_q\epsilon_{t-q}$ 
\item Here $\alpha _{i}$ are the parameters of the autoregressive part of the model, $\theta _{i}$ are the parameters of the moving average part, $d$ is the degree of differencing and $\epsilon _{t}$ are error terms.  
\end{itemize}

\end{itemize}
}



\frame{\frametitle{Machine Learning Models I} 
\begin{itemize}
\item Least Absolute Shrinkage and Selection Operator
\begin{itemize}
\item The LASSO is a penalized version of the OLS: 
\begin{equation*}
\min_{\beta} { ||X \beta - y||_2 ^ 2 + \alpha ||\beta||_1 }
\end{equation*}
\end{itemize}


\ \\

\item Support Vector Regression
\begin{itemize}
\item The SVR is an adapted version of a SVM for regression problems and tries to solve the optimization problem: 
  \begin{equation*}
\min_{\beta} { \frac{1}{2} || \beta ||_2 ^ 2 } \text{, subject to  } || X \beta - y || < \varepsilon
\end{equation*}
\end{itemize}

\end{itemize}
}



\frame{\frametitle{Machine Learning Models II} 
\begin{itemize}
\item Regression Tree
\begin{itemize}

\item Binary tree that groups data with similar vaules into the same leaf. 
\item This is done to minimize the overall loss: 
\begin{equation*}
L(...
\end{equation*}

\end{itemize}

\ \\

\item Gradient Booster
\begin{itemize}
\item Ensemble of the from \begin{equation*}
f(x) = \sum_{i = 1}^N f_i(x)
\end{equation*} where $f_i$ are weak learners, most of the time tree based models. 
\item Are called gradient booster because of the way the model is trained.
\end{itemize}


\end{itemize}
}







\frame{\frametitle{Deep Learning} 


\begin{itemize}
\item  Reccurent Neural Network


\begin{itemize}
\item A RNN is a deep neural network that is designed to handle sequential data. 
\item A RNN cell is defined as:
\begin{equation*}
h_t = \sigma(W_{ih} x_t + b_{ih} + W_{hh} h_{(t-1)} + b_{hh})
\end{equation*}
\item There are also more sophisticated approched like the LSTM (Long short-term memory). 
\end{itemize}
\end{itemize}


}







\subsection{Evaluation}
\frame{\frametitle{Evaluation} 
Noch Schreiben. Erkärung des signifikanz tests
}



\section{Results} 
\subsection{Training on single countries}
\frame{\frametitle{Results - Training on single countries}
Tabelle hier
}

\frame{\frametitle{Results - Training on single countries} 
\begin{figure}[htbp] 
\centering
\includegraphics[scale=0.65]{Germany.png}   
      % \caption{Simulationsumgebung RoboDK.}
\end{figure}
}


\subsection{Training on whole dataset countries}
\frame{\frametitle{Results - Training on single countries}
Tabelle hier
}

\frame{\frametitle{Results - Training on single countries} 
\begin{figure}[htbp] 
\centering
\includegraphics[scale=0.65]{Germany.png}   
      % \caption{Simulationsumgebung RoboDK.}
\end{figure}
}


\subsection{Test for statistical significance}
\frame{\frametitle{Test for statistical significance}
\begin{figure}[htbp] 
\centering
\includegraphics[scale=0.65]{result.png}   
      % \caption{Simulationsumgebung RoboDK.}
\end{figure}
}



\section{Conclusion} 
\frame{\frametitle{Conclusion}

\begin{itemize}
\item The "expert-based" decesion should be derived based on data
\end{itemize}



}



\end{document}
