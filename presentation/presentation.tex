
\documentclass[mathserif]{beamer}

\setbeamertemplate{navigation symbols}{}%remove navigation symbols
\setbeamercolor{itemize item}{bg=gray, fg=gray} 
\setbeamercolor{itemize subitem}{fg=gray}
\setbeamercolor{section in toc}{fg=gray}
\setbeamercolor{subsection in toc shaded}{fg=gray}
\setbeamercolor{title}{fg=black}


\setbeamercolor{frametitle}{fg=white,bg=black}
%\setbeamertemplate{frametitle}[default][left]


\setbeamertemplate{frametitle}
{%
\vspace{-0.165ex}

\begin{beamercolorbox}[wd=\paperwidth,dp=1ex, ht=4.5ex, sep=0.5ex, colsep*=0pt]{frametitle}%
    \usebeamerfont{frametitle}   \strut \insertframetitle  \hfill \raisebox{-2ex}[0pt][-\ht\strutbox ]{ \includegraphics[scale=0.23]{Gestaltung/cfds_logo.png}} 
    \end{beamercolorbox}%
 }%
 
 
 
 
 \setbeamertemplate{footline}
{
  \leavevmode%
  \vspace*{0.2cm}
     \hspace*{-2cm}
	\color{black}\rule{15cm}{0.2mm}   
  \hbox{%
  \begin{beamercolorbox}[wd=1.15\paperwidth,ht=1.5ex,dp=1ex, center]{author in head/foot}%
    \usebeamerfont{author in head/foot}\color{black} CFDS Project \hfill  Slide \insertframenumber /\inserttotalframenumber
\vspace*{0.02cm}
  \end{beamercolorbox}}%
}



%\usetheme{default}


%\setbeamercolor{frametitle}{fg=black}


\setbeamertemplate{navigation symbols}{}%remove navigation symbols
\setbeamercolor{itemize item}{bg=gray, fg=gray} 
\setbeamercolor{itemize subitem}{fg=gray}
\setbeamercolor{section in toc}{fg=gray}
\setbeamercolor{subsection in toc shaded}{fg=gray}
\setbeamercolor{title}{fg=black}

\setbeamercolor{section in head/foot}{fg=white, bg=black}

%\setbeamertemplate{frametitle}
%{
%\includegraphics[width=3cm]{Gestaltung/oth_logo.png}\hfill
%%\parbox{0.6\textwidth}{\color{black}
%%\vspace*{-1cm}{
%\insertframetitle
%%}
%%}
%\hspace*{-0.3cm}\color{black}\rule{20cm}{0.2mm}
%}

\setbeamertemplate{frametitle}
{% 
%\vspace{-0.165ex} 
\begin{beamercolorbox}[wd=\paperwidth,dp=1ex, ht=7ex, sep=0.5ex, colsep*=0pt]{section in head/foot}% 
    \usebeamerfont{frametitle}   \strut 
    \raisebox{-0.55ex}[0pt][-\ht\strutbox ]{
        \includegraphics[width=3.5cm]{Gestaltung/cfds_logo.png}
    	}  
    \hfill  \insertframetitle \hspace{0.4cm}
    \end{beamercolorbox}
    \vspace*{-0.5cm}
    \hspace*{-2cm}
\color{black}\rule{15cm}{0.2mm}               
}%


%\setbeamertemplate{headline}{
%\begin{beamercolorbox}{frametitle}


%\begin{flushright}

%\end{flushright}
%}} 
%\end{beamercolorbox}%

%}
\setbeamertemplate{footline}
{
  \leavevmode%
  \vspace*{0.2cm}
     \hspace*{-2cm}
	\color{black}\rule{15cm}{0.2mm}   
  \hbox{%
  \begin{beamercolorbox}[wd=1.15\paperwidth,ht=1.5ex,dp=1ex, center]{author in head/foot}%
    \usebeamerfont{author in head/foot}\color{black}\insertshortauthor \hfill
CFDS \hfill    Slide \insertframenumber /\inserttotalframenumber
\vspace*{0.02cm}
  \end{beamercolorbox}}%
}

\setbeamerfont{tiny structure}{size=\tiny}


\defbeamertemplate*{title page}{customized}[1][]
{
\begin{minipage}{0.8\linewidth}
  \usebeamerfont{title}\inserttitle\par
  \usebeamerfont{subtitle}\usebeamercolor[fg]{subtitle}\insertsubtitle\par
  \bigskip
  \usebeamerfont{author}\insertauthor\par
  \usebeamerfont{institute}\insertinstitute\par
  \usebeamerfont{date}\insertdate\par
  %\usebeamercolor[fg]{titlegraphic}\inserttitlegraphic
\end{minipage} \hfill
\begin{minipage}{0.15\linewidth}
  \vspace*{0pt}
  \usebeamercolor[fg]{titlegraphic}\inserttitlegraphic
\end{minipage}
  
  
}

%\AtBeginSection[]{
%  \begin{frame}
%  \vfill
  
  
  
%  \includegraphics[width=3.5cm]{Gestaltung/cfds_logo.png}  
  
%  \centering
%  \begin{beamercolorbox}[sep=8pt,center,shadow=true,rounded=true]{title}
%    \usebeamerfont{title}\insertsectionhead\par%
%  \end{beamercolorbox}
%  \vfill
%  \end{frame}
%}



\begin{document}




%\title{

%CFDS06 Project


%\ \\

% Macroeconomic forecasting: Can machine learning methods outperform traditional approaches?}   
%\author{Felix Jobson} 
%\date{\today} 


%%\titlegraphic{\includegraphics[width=3.5cm]{Gestaltung/cfds_logo.png}}
%%\institute{asdf}


%\frame{\titlepage} 








\title{\textbf{Macroeconomic forecasting: Can machine learning methods outperform traditional approaches?}}
\subtitle{\ \\CFDS06 Project}
%\institute{Fakultät für Informatik und Mathematik\\ OTH Regensburg}

\author{Felix Jobson
\ \\}
\date{\ \\ 29.05.2021} 
%\date{\today}
%\begin{document} 

{
\setbeamertemplate{footline}{} 
\begin{frame}
\frametitle{~}
\maketitle 
\end{frame}
}
\addtocounter{framenumber}{-1}









\frame{\frametitle{Table of contents}\tableofcontents} 

{
\setbeamertemplate{footline}{} 
\section{Problem Description} 
\frame{\frametitle{~}

\centering \huge Problem Description

}
}
\addtocounter{framenumber}{-1}



 
\frame{\frametitle{Problem Description}
\begin{itemize}
\item The research question of the project is the capability of machine learning models to predict the growth of an economy and compare the result with traditional methods of forecasting.

\ \\

\item The dependent variable is the growth rate of the gross domestic product (GDP). This is the
objective of the learning and prediction task. The independent variables are several
macroeconomic factors.

\ \\

\item The baseline models are classical econometric methods and the World Economic Outlook of the International Monetary Fund.



\end{itemize}
}


{
\setbeamertemplate{footline}{} 
\section{Data} 
\frame{\frametitle{~}

\centering \huge  Data

}
}
\addtocounter{framenumber}{-1}






\subsection{Sources and Overview}
\frame{\frametitle{Data sources}
\begin{itemize}

\item Sources
\begin{itemize}
\item International Monetary Fund (IMF)
\item Organisation for Economic Co-operation and Development (OECD)
\end{itemize}

\ \\

\item Time Period: 1980 - 2017
\begin{itemize}
\item Training Set: 1980 - 2004
\item Validation Set: 2005 - 2010
\item Test Set: 2011 - 2017 
\end{itemize}

\ \\

\item Countries:
\begin{itemize}
\item Initially 189
\item After clearning 46
\end{itemize}

 


\end{itemize}
}



\frame{\frametitle{Variables}

\begin{itemize}

\item Number of macroeconomic factors used: 
\begin{itemize}
\item Initial: 41
\item After cleaning 15
\end{itemize}


\ \\

\item Examples of used varibles 
\begin{itemize}
\item Inflation
\item Unemployment rate
\item Material consumption
\item Working age population
\item Fertility rates

\end{itemize}


\end{itemize}


}


\frame{\frametitle{Split of the dataset}
 \begin{itemize}
 \item Two different purposes: 
 
 \ \\
 
 \item Model selection and model assessment.

 
 \ \\

 \item The validation set is used to estimate the prediction error for model selection.
 
 \ \\
 
 \item The test set should be kept in a "vault" and is used to estimte the test error at the end of the analysis. 
 


 \end{itemize}
}



\subsection{Preparation of the Data}
\frame{\frametitle{Transforming to Growth Rates}
\begin{itemize}
\item Because the variable have different absolute values, growth rates are used. 

\ \\

\item To receive the same magnitude for an increase as well as a decrease a logarithmic transformation is used: 

\ \\

\item 
\begin{equation*}
\hat{x}_{i} = \ln(\frac{x_{i}}{x_{i-1}} + |\min_j(x_{j})| + 0.001)
\end{equation*}
\end{itemize}
}




\frame{\frametitle{Preparation of the Data}
 \begin{itemize}
\item Using the framework of supervised learning to work with time series.

\ \\

\item The orignal data is given in the form $(x_t, y_t)$, $t = 1 ... N$

\ \\

\item For every time step the outcome $y$ is mapped to predictor variables $x$ that are preceeding:  \begin{equation*}
(x_{t-1}, y_t), ~ t = 2 ... N
 \end{equation*}
 \item Hence a model for supervised learning can be trained and used for prediction. 
 \end{itemize}
}


\subsection{Missing Values}
\frame{\frametitle{Impute missing values}
 \begin{itemize}
 \item Only countries with less than 50 \% missing values are used. Then the top 15 filled variables are selected.  
 
 \ \\

 \item To use time series with missing data at all, an imputing strategy is used: \textit{k-nearest neighbors} 
 
\ \\

 
 \item Each sample's missing values are imputed using the mean value from n nearest neighbors found in the training set.
 
 \ \\
 
 \item Important: Fit on the training set and then apply imputation on the validation and test set.  
 


 \end{itemize}
}







{
\setbeamertemplate{footline}{} 
\section{Approach} 
\frame{\frametitle{~}

\centering \huge Approach

}
}
\addtocounter{framenumber}{-1}



\subsection{Models}


\frame{\frametitle{World Economic Outlook } 
\begin{itemize}
\item The International Monetary Fund publishes predictions of the GDP growth in its World Economic Outlook (WEO)

\ \\

\item The IMF publishes the WEO twice a year in spring and fall.

\ \\

\item The prediction from the fall is used, as this is closer to the next year and therefore the prediction is more precise.
\end{itemize}
}



\frame{\frametitle{Classic Models} 
\begin{itemize}
\item Ordinary Least Squares
\begin{itemize}
\item The OLS regression is the most famous and basic model in econometrics. It has the following form: 
\begin{equation*}
y = x_1\beta_1 + x_2\beta_2 + ... x_N\beta_N + \beta_{N+1}
\end{equation*}
\end{itemize}

\ \\

\item Autoregressive Integrated Moving Average
\begin{itemize}
\item The autoregressive integrated moving average ARIMA($p$, $d$, $q$) model is used in time series analysis. 
\item $X_t - \alpha_1X_{t-1} - ... - \alpha_pX_{t-p} =  \epsilon_t + \theta_1\epsilon_{t-1} + ... + \theta_q\epsilon_{t-q}$ 
\item Here $\alpha _{i}$ are the parameters of the autoregressive part of the model, $\theta _{i}$ are the parameters of the moving average part, $d$ is the degree of differencing and $\epsilon _{t}$ are error terms.  
\end{itemize}

\end{itemize}
}



\frame{\frametitle{Machine Learning Models I} 
\begin{itemize}
\item Least Absolute Shrinkage and Selection Operator
\begin{itemize}
\item The LASSO is a penalized version of the OLS: 
\begin{equation*}
\min_{\beta} { ||X \beta - y||_2 ^ 2 + \alpha ||\beta||_1 }
\end{equation*}
\end{itemize}


\ \\

\item Support Vector Regression
\begin{itemize}
\item The SVR is an adapted version of a SVM for regression problems and tries to solve the optimization problem: 
  \begin{equation*}
\min_{\beta} { \frac{1}{2} || \beta ||_2 ^ 2 } \text{, subject to  } || X \beta - y || < \varepsilon
\end{equation*}
\end{itemize}

\end{itemize}
}



\frame{\frametitle{Machine Learning Models II} 
\begin{itemize}
\item Regression Tree
\begin{itemize}

\item Binary tree that groups data with similar vaules into the same leaf. The response in each leaf $L_1, L_2, ..., L_M$ is modeled as constant, so the tree can be expressed as a function: 
\begin{equation*}
f(x) = \sum_{i = 1}^M c_mI(x \in L_m)
\end{equation*}

\end{itemize}

\ \\

\item Gradient Booster
\begin{itemize}
\item Ensemble of the from \begin{equation*}
f(x) = \sum_{i = 1}^N f_i(x)
\end{equation*} where $f_i$ are weak learners, most of the time tree based models. 
\item Are called gradient booster because of the way the model is trained.
\end{itemize}


\end{itemize}
}







\frame{\frametitle{Deep Learning} 


\begin{itemize}
\item  Recurrent Neural Network


\begin{itemize}
\item A RNN is a deep neural network that is designed to handle sequential data. 
\item A RNN cell is defined as:
\begin{equation*}
h_t = \sigma(W_{ih} x_t + b_{ih} + W_{hh} h_{(t-1)} + b_{hh})
\end{equation*}
\item There are also more sophisticated approches like the LSTM (Long short-term memory). 
\end{itemize}
\end{itemize}


}



\subsection{Evaluation}
\frame{\frametitle{Evaluation} 
\begin{itemize}

\item The performance of the models is measured by the MSE of the test set: 
\begin{equation*}
MSE = \frac{1}{|T|} \sum_{t \in T} (y_t - \hat{y}_t)^2
\end{equation*}

\ \\ 


\item The cartesian  product $\Omega$  of the set of all classical models and all machine learning models is formed. The MSE of both is compared:
\begin{equation*}
X(\omega) :=\begin{cases}1& \text{if  } MSE_{ML_{\omega}} > MSE_{classic_{\omega}} \\
 0 & \text{else} \end{cases}  ~~~  \omega \in  \Omega
\end{equation*} 

%\item The number of time, when the MSE of the machine learning model is less then the MSE of the classical model 



 
\end{itemize}
}



\frame{\frametitle{Evaluation} 

\begin{itemize}



\item Confidence intervals of $X$ are approximated by bootstrapping. 

\ \\ 

\item If the lower bound of the confidence interval is greater than 0.5, the machine learning methods have statistically significant better performance than the traditional approaches.

\ \\ 

\item Evaluating this approach with two different settings:

\ \\ 

\item Training each model with the data of a single country.

\ \\ 

\item Training each model with the whole data combined. 

 

\end{itemize}

}



{
\setbeamertemplate{footline}{} 
\section{Results} 
\frame{\frametitle{~}

\centering \huge Results

}
}
\addtocounter{framenumber}{-1}






\frame{\frametitle{Deep Learning went wrong}
 
\begin{figure}[htbp] 
\centering
\includegraphics[scale=0.34]{rnn_optim.png}   
      % \caption{Simulationsumgebung RoboDK.}
\end{figure}
}

\frame{\frametitle{Result Training Single Country}
\begin{figure}[htbp] 
\centering
\includegraphics[scale=0.75]{result_single_models.png}   

\end{figure}
}


\frame{\frametitle{Result Training All Countries}
\begin{figure}[htbp] 
\centering
\includegraphics[scale=0.75]{result_all_models.png}   

\end{figure}
}



\frame{\frametitle{Test for statistical significance}
\begin{figure}[htbp] 
\centering
\includegraphics[scale=0.75]{result.png}   

\end{figure}
}



{
\setbeamertemplate{footline}{} 
\section{Conclusion}
\frame{\frametitle{~}

\centering \huge Conclusion

}
}
\addtocounter{framenumber}{-1}


\frame{\frametitle{Conclusion I}

\begin{itemize}
\item Machine learning models can outperform traditional approaches!

\ \\

\item At least in the given evaluation framework presented.

\ \\

\item Data collection and handling take the most time from the project budget, modelling takes only a fraction. 

\ \\

\item Deep learning relies heavily on the amount of data and fails if there is not enough available.  

 
\end{itemize}

}

\frame{\frametitle{Conclusion II}

\begin{itemize}

\item Even simple machine learning models have a decent performance.

\ \\

\item SVR failed on training with all countries. A profound understanding of the model is important to understand problems. 

\ \\ 

\item The proposed deep reinforcement learning approach was not successful. 


\end{itemize}

}

 
\frame{\frametitle{Possible Next Steps}

\begin{itemize}
\item Collect better Data in terms of quality and quantity. 

\ \\

\item The "expert-based" decision should be derived based on data.

\ \\

\item Analyse feature importance and automate feature selection. 

\ \\ 

\item Analyse the transformation of the data and use a more sophisticated approach. 

\end{itemize}

}



{
\setbeamertemplate{footline}{} 
\frame{\frametitle{~}

\centering \huge "All models are wrong, but some are useful"

}
}
\addtocounter{framenumber}{-1}



\end{document}


