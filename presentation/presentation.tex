
\documentclass{beamer}
\begin{document}
\title{CFDS06 Project}   
\author{Felix Jobson} 
\date{\today} 

\frame{\titlepage} 

\frame{\frametitle{Table of contents}\tableofcontents} 



\section{Data} 
\subsection{Sources and EDA}
\frame{\frametitle{Data sources}

}

\frame{\frametitle{EDA of whole data}

}

\subsection{Split of the data set}
\frame{\frametitle{Split of the data set}
 
}

\subsection{Missing Values}
\frame{\frametitle{Filitering missing values}

}

\frame{\frametitle{Impute missing values}
 
}


\frame{\frametitle{Preparation of the Data}
 \begin{itemize}
 \item Using the framework of supervised learning to work with time series. This approach is used in (Reference to Crystal-Ball)
 \item The orignal data is given in the form $(x_t, y_t)$, $t = 1 ... N$
 \item For every time step the outcome $y$ is mapped to predictor variables $x$ that are preceeding:  \begin{equation*}
 (x_{t-1}, y_t), ~ t = 2 ... N
 \end{equation*}
 \item Hence a model for supervised learning can be trained and used for prediction. 
 \end{itemize}
}





\section{Models} 

\subsection{World Economic Outlook}
\frame{\frametitle{World Economic Outlook } 
\begin{itemize}
\item The International Monetary Fund publishes predictions of the GDP growth in its World Economic Outlook (WEO)
\item The IMF publishes the WEO twice a year in spring and in fall
\item I will use the prediction of the fall, as this closer to the next year and therefore the prediction should be more precise
\end{itemize}
}


\subsection{Linear Model}
\frame{\frametitle{Linear Model} 
\begin{itemize}
\item The ordinary least squares regression is the most famous and basic model in econometrics. It has the following form: 
\item \begin{equation*}
y = x_1\beta_1 + x_2\beta_2 + ... x_N\beta_N + \beta_{N+1}
\end{equation*}
\end{itemize}
}



\subsection{ARIMA}
\frame{\frametitle{ARIMA} 
\begin{itemize}
\item The autoregressive integrated moving average ARIMA($p$, $d$, $q$) model is used in time series analysis. 
\item $X_t - \alpha_1X_{t-1} - ... - \alpha_pX_{t-p} =  \epsilon_t + \theta_1\epsilon_{t-1} + ... + \theta_q\epsilon_{t-q}$ 
\item Here $\alpha _{i}$ are the parameters of the autoregressive part of the model, $\theta _{i}$ are the parameters of the moving average part, $d$ is the degree of differencing and $\epsilon _{t}$ are error terms.  
\item There is an implementaion in python in the pmdarima package, which automatically discovers the optimal order for an ARIMA model with exogenous variables. 
\end{itemize}
}


\subsection{Gradient Booster}
\frame{\frametitle{Gradient Booster} 
\begin{itemize}
\item Currentlty the best "from the shelf" models for tabular data. 
\item Ensemble of the from \begin{equation*}
f(x) = \sum_{i = 1}^N f_i(x)
\end{equation*} where $f_i$ are weak learners, most of the time tree based models. 
\item Are called gradient booster because of the way the model is trained. 
\item Famous implementations: Xgboost, light bgm, cat boost. 
\end{itemize}
}



\subsection{RNN}
\frame{\frametitle{RNN} 
asdf
}

\subsection{Deep RL I}
\frame{\frametitle{Deep RL I} 
asdf
}

\subsection{Deep RL II}
\frame{\frametitle{Deep RL II} 
asdf
}




\section{Results} 
\subsection{Training on single countries}
\frame{\frametitle{Results - Training on single countries}
Tabelle hier
}

\frame{\frametitle{Results - Training on single countries} 
\begin{figure}[htbp] 
\centering
\includegraphics[scale=0.65]{Germany.png}   
      % \caption{Simulationsumgebung RoboDK.}
\end{figure}
}


\subsection{Training on whole dataset countries}
\frame{\frametitle{Results - Training on single countries}
Tabelle hier
}

\frame{\frametitle{Results - Training on single countries} 
\begin{figure}[htbp] 
\centering
\includegraphics[scale=0.65]{Germany.png}   
      % \caption{Simulationsumgebung RoboDK.}
\end{figure}
}


\subsection{Test for statistical significance}
\frame{\frametitle{Test for statistical significance}
\begin{figure}[htbp] 
\centering
\includegraphics[scale=0.65]{result.png}   
      % \caption{Simulationsumgebung RoboDK.}
\end{figure}
}







\end{document}
